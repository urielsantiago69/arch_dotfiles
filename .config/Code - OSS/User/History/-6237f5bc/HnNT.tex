\documentclass[../main.tex]{subfiles}

\begin{document}
    
\section{What is administrative law?}
The Constitution defines the mechanisms of government along three principal branches: Legislative, executive, and judicial. Most of your direct interactions with government, however, will not be with any of these main branches, but with a variety of administrative agencies. Often, these agencies are part of the executive branch, but are created by the legislature and tasked with implementing laws in all its myriad, often ordinary details.

 \enquote{Administrative law is concerned with the legal problems arising out of the existence of agencies which combine in a single entity legislative, executive, and judicial powers. Acts necessary to carry out legislative policies and purposes already declared by the legislature are administrative.}\footnote{2 Am. Jur. 2d Administrative Law § 1 (2023)}

\begin{note}
    \textbf{Most modern administrative agencies are characterized by the combination of judicial power (adjudication) and legislative power (rulemaking). Nevertheless, agencies are still funded through appropriations from Congress, which can exercise oversight powers. The executive appoints the agency's personnel. The judiciary retains the residual power to check agency action.}
\end{note}

\begin{shaded}

    \textbf{Example:} The Government Service Insurance Service (GSIS)is an administrative agency created by Congress (through Republic Act No. 8291). The law tasked it with the management of an insurance and pension system for government employees. The operation of an insurance system and its funds can be a complex task, and the law did not have a lot of details as to how. What are some of the things required for this task? The GSIS would have to know its members - to identify them and contact them (e.g. for premium payments due).To do that, there has to be a system for updating their contact info. The GSIS created a rule, GSIS Memorandum Circular No. 045 s.2022, applicable to government agencies and GSIS members, which provides a new process for updating contact information. This can affect any government employee. If he has to move, he has to go through this process in order to preserve his benefits. In this case, the GSIS is exercising its rulemaking power. On the other hand, if the government employee is injured and has to make an insurance claim, the GSIS can provide the payout as long as the legal conditions to do so (i.e. actual injury) are present. The GSIS is simply implementing the law, carrying out its executive function. However, if there is a dispute regarding the necessity of a payout, or its amount, there is a process for escalating the matter to the GSIS board, which can receive and evaluate both supporting evidence for factual claims as well as legal arguments. This is an instance of the GSIS exercising its adjudicatory power(?) \newline

    \textit{Is there such a thing as pure executive power that will not involve some form of adjudication or rulemaking? Isn't there some rule interpretation, as well as evidentiary evaluation in the initial approval or denial of the claim? These functions all intersect. We have not made an analysis of the content of these functions, and simply distinguished them based on who is performing them - which is almost tautological.}
\end{shaded}

\subsection{The administrative state}Development of the “administrative state” - as a way to cope with growing complexity, scale, and pace. So many of our administrative agencies were established to deal with subjects of modernity and the industrial revolution: Mass transportation and telecommunications, the need for urban housing, more complicated financial transactions. I would add to the Court's characterization of these problems as not only differentiated in terms of complexity, but in terms of risk. Even before the administrative state people have had transportation, but our modern-day cars - the thousands of multi-ton chunks of metal that can travel at least 60kph (and loaded with gasoline!) are orders of magnitude riskier than horse-drawn carriages. Think about how the sense of scale and connectedness of factories and stock exchanges increases the risk of environmental and economic collapse.


\begin{shaded}
    \textbf{Traditional models of governance can't scale} Theoretically, we could have dealt with all these using classical legal models: Either a contract regime (based on agreement, you will pay me money if I allow you to build a factory next to my house) or liability regime (based on a law passed by Congress you pay me money if your car hits me). In both cases, we can rely on the courts (and the police, under the executive) for enforcement. That has always worked somewhat during the pre-industrial period. However, the model is strained because of the nature and the degree of risk involved with these modern problems. There is no way for me to contract in advance with every car-owner, and the court (at least prior to class action systems) would be hard-pressed to determine the liability of a massive polluter on a case-by-case basis.

    Congress can’t do administrative rulemaking. It’s not really built for speed, efficiency but extended deliberation. It has limited session days. Requires coordination of two, multi-member bodies (vulnerable to collective action failure). Two chambers, three readings (assuiming it has gone through committee). Congress is not the domain of experts and politicians.
\end{shaded}

Much of what administrative agencies do is formulate and enforce regulations that manage risks ex-ante, to prevent or mitigate the risks. The LTO will not adjudicate liability for a car crash. It will, however enforce requirements before you are allowed to drive a car in the first place - passing a driver's test, having third party liability insurance, the occasional checks to ensure that your vehicle is still roadworthy. And of course, every car should have a license plate, which makes attribution of liability easier (and incentivizes drivers to be safer on the road). Both Congress and the Judiciary do not have the experience and institutional design for this kind of regulation as a permanent concern. Can you imagine Congress or the courts running a vehicle registration system? A drugs approval system? Thus the need for specialized agencies with the necessary expertise.

\textit{Should administrative agencies validly exist in the first place, given separation of powers and the non-delegation doctrine? Why are they necessary?}

 
\subsection{The non-delegation doctrine}

\subsubsection{Early forms of delegation} Specific problem for the king (such as raising an army, collecting revenue, handling negotiations) were tasked to a trusted lord. Some of these arrangements were formalized into Great Offices - Chancellor, Chamberlain, Seneschal, with some degree of permanence. However, there was less formalism when it comes to intermixing of offices and functions. 

Even under early constitutional governments: Before, three branches, some historical offices or commissions under Congress or the executive usually enough to deal with the business of governance (?). Occasional delegation of specific, time bound matters to commissions and executive.

\subsubsection{Basis and rationale} The non-delegation doctrine is not expressly provided in the Constitution. But it is implied from structural separation of powers, and the model of the government as agents for the public good. The doctrine provides that \textbf{Congress cannot further delegate the legislative power that has been delegated to it by the sovereign people}. This is based on traditional agency principle: \textit{Delegata potesta non potest delegare}:"What has been delegated cannot be delegated (further)".\footnote{\enquote{The Legislative cannot transfer the Power of Making Laws to any other hands. For it being but a delegated Power from the People, they, who have it, cannot pass it over to others. ... And when the people have said, We will submit to rules, and be govern'd by Laws made by such Men, and in such Forms, no Body else can say other Men shall make Laws for them; nor can the people be bound by any Laws but such as are Enacted by those, whom they have Chosen, and Authorised to make Laws for them. The power of the Legislative being derived from the People by a positive voluntary Grant and Institution, can be no other, than what the positive Grant conveyed, which being only to make Laws, and not to make Legislators, the Legislative can have no power to transfer their Authority of making laws, and place it in other hands (John Locke, Second Treatise of Civil Government (1690))}}

Having been given tremendous powers by the sovereign, it is a form of disloyalty and neglect to then delegate these powers further - especially to unelected bureaucrats that are not subject to constitutional checks. The prohibition preserves a sense of accountability for the legislature as a political branch - The people voted for them to set policy, and they cannot evade democratic accountability by "passing the buck".


\subsubsection{Validity based on non-legislative nature of power}
%These need to be paraphrased because I am lifting them directly from other annotations 

\textbf{Enforcement} - Congress can delegate the power to enforce a law. This involves exercise of discretion as to whether the law should apply to individual cases. \textit{Bowsher v. Synar}\footnote{478 U.S. 714 (1986)} The Court held that the delegation was valid because the Secretary’s authority was limited to determining whether the state plan complied with the Act’s requirements, and the Secretary was required to make his determination in accordance with the Act’s standards. The Court also held that the Secretary’s authority was not legislative in nature because the Secretary was not empowered to make policy decisions.  

\textit{In this case, is Congress really delegating or specifying the delegate? After all, the Executive has constitutional power and duty to enforce the law. Can Congress have properly done it itself - by passing more specific legislation?}


\textbf{Subordinate legislation} - Congress cannot delegate its power to make laws, but can delegate the power to fill in the details. \textit{Wayman v. Southard}\footnote{23 U.S. (10 Wheat.) 1, 41 (1825).} In this case Congress had delegated to the courts the power to prescribe judicial procedure; it was contended that Congress had thereby unconstitutionally clothed the judiciary with legislative powers. While Chief Justice John Marshall conceded that the determination of rules of procedure was a legislative function, he distinguished between "important" subjects and mere details. Marshall wrote that "a general provision may be made, and power given to those who are to act under such general provisions, to fill up the details."Justice Marshall:  Although Congress may not delegate powers that “are strictly and exclusively legislative,” it may delegate “powers which [it] may rightfully exercise itself.”


\textbf{Contingent legislation} - Congress can delegate the power to make laws contingent on ascertaining facts that bring the law into operation. Note that this is not about applying a general statute to individual cases, or supplementing it through detailed regulation. This concerns whether a statute will be revived, suspended, modified, or that a new rule be put into operation based on factual findings of the executive or administrative officer. \textit{Field v. Clark}\footnote{143 U.S. 649 (1892)} Tariff Act of 1890 was assailed as unconstitutional because it directed the President to suspend the free importation of enumerated commodities “for such time as he shall deem just” if he found that other countries imposed upon agricultural or other products of the United States duties or other exactions that “he may deem to be reciprocally unequal and unjust.” In sustaining this statute the Court relied heavily upon two factors: (1) legislative precedents, which demonstrated that “in the judgment of the legislative branch of the government, it is often desirable, if not essential,...to invest the President with large discretion in matters arising out of the execution of statutes relating to trade and commerce with other nations,” and (2) that the act did “not, in any real sense, invest the President with the power of legislation...Congress itself prescribed, in advance, the duties to be levied,...while the suspension lasted. Nothing involving the expediency or the just operation of such legislation was left to the determination of the President...He had no discretion in the premises except in respect to the duration of the suspension so ordered.”


\textit{What is the difference between contingent legislation and enforcement?}

\begin{shaded}

The analytical framework is based on making distinctions between the nature of the delegated power, and the core legislative power exercised by Congress. Legislation is about determining the law, while the Judiciary is concerned with interpreting the law (for specific cases), and the executive is to just implement the law. The framework assumes that there are such things as pure legislative power, as well as a pure judicial power, from which quasi legislative and quasi judicial powers are distinguished. Quasi-legislative and quasi-judiciary categories and their location in the executive reveal the inherent flaws from that division. Very rarely is an act purely legislative, judicial, or executive.

Let’s go through a thought experiment. Imagine that Congress passes a law which says that “No dogs. cats, or cows, or other animals are allowed in the park.” And delegates implementation of this law to the Parks Authority, an agency in the executive branch. Is there anything about the content or substance of legal statements that make them necessarily, inherently legislative, judicial, executive? I would submit there is none.

1. If the executive finds a dog and removes it - Then you can say that this is an act of enforcement, since it only acting based on clear, unambiguous instruction of the text of the law. There is no interpretation, no filling in of details involved
    1. But what if instead of public enforcement, we have a regime of private enforcement, where instead of public authority enforcing the ban, citizens have to sue in court?
    2. Could also be true if the law requires that every removal be through a court order. Interpretation is implied in execution (even for clear laws)
    3. Doesn’t this execution now have the character of judicial determination - an interpretation based on plain meaning is still a judicial act
    
2. What if the executive removes a crocodile? Not provided in the enumeration but still within “other animals”? Isn’t this now a form of interpretation usually reserved for the judiciary? What if it’s a bee - and this really isn’t what the law meant - and this is subject to judicial review by the court. Doesn’t this mean that what the park authority did was judicial in the first place? Court could rule that: 1. The law only meant to cover domesticated animals, or wild terrestial animals, or only four-legged animals.
3. But: Couldn’t Congress just have provided for all this in the first place? If it provided the same information - i.e., that crocodiles are included or bees are excluded. Isn’t that legislative in nature? Congress can do specific legislation all the time. Sometimes it does legislation that

Specificity, or any attribute of the text or its meaning is not really the defining criteria that will separate legislative power, from executive and judicial. It’s all just imperative knowledge. The only operational criteria seems to be: who is exercising the power (or creating, processing the imperative knowledge). In that case - won’t the definition of “legislative power” etc be tautological? And all delegation will not be allowable? How do we measure and delineate levels of specificity in the first place, and assign only some levels to some branches.

\end{shaded}

\textbf{Local governments} Historically, and constitutionally, local governments have had legislative bodies that can pass laws with local application (as long as they are consistent with national law)

\textit{Calalang v. Williams}\footnote{G.R. No. 47800. December 2, 1940.} The Philippine Supreme Court sustained Section 1 of Commonwealth Act No. 548, which authorize the Secretary of Public Works and Communications to promulgate rules and regulations for the enforcement of the provisions of the Act. The Court held that the Secretary of Public Works and Communications was not exercising legislative power, but a form of enforcement:

\begin{displayquote}
The delegated power, if at all, therefore, is not the determination of what the law shall be, but merely the ascertainment of the facts and circumstances upon which the application of said law is to be predicated.To promulgate rules and regulations on the use of national roads and to determine when and how long a national road should be closed to traffic, in view of the condition of the road or the traffic thereon and the requirements of public convenience and interest, is an administrative function which cannot be directly discharged by the National Assembly. It must depend on the discretion of some other government official to whom is confided the duty of determining whether the proper occasion exists for executing the law. But it cannot be said that the exercise of such
discretion is the making of the law.
\end{displayquote}

In citing language used to describe enforcement and contingent legislation to justify subordinate legislation, the Court conflates all concepts into \enquote{administrative function}.

\subsection{Validity based on sufficient standards}

First approach above: Is to say that it's not really core legislative function. That its enforcement, or filling in details or simply determining factual basis for operation of the law. All these supposedly fall short of the important policy calls that only Congress can make. Problem is that: (1) Congress can in fact, through specific legislation, achieve the same ends; (2) Important policy can in fact arise from the agency's exercise of delegated powers. The  second more recent approach is to make judicial acceptance of delegated power contingent on the presence and sufficiency of standards. 

Purpose of the standard requirement: (1) The fundamental policy decisions are not made by mere appointed officials and (2) Provides a basis for judicial review.\footnote{It insures that the fundamental policy decisions in our society will be made not by an appointed official but by the body immediately responsible to the people..., [and] it prevents judicial review from becoming merely an exercise at large by providing the courts with some measure against which to judge the official action that has been challenged.”, \textit{Arizona v. California}, 373 U.S. 546, 626 (1963) Justice Harlan, dissenting.}


\textbf{Intelligible principles} - A delegation of legislative power is valid as long as Congress provides for an ``intelligible principle''. \textit{J. W. Hampton, Jr. \& Co. v. United States}\footnote{276 U.S. 394 (1928)} The U.S. Supreme Court upheld Congress’s delegation to the President of the authority to set tariff rates that would equalize production costs in the United States and competing countries.  In seeking the cooperation of another branch Congress was restrained only according to “common sense and the inherent necessities”. Court would sustain delegations whenever Congress provided an “intelligible principle” to which the President or an agency must conform. The terminology would later evolve to "sufficient standard".

\textbf{Completeness} In \textit{Pelaez v. Auditor General}\footnote{15 SCRA 569 (1965)} Philippine Supreme Court struck down the President's power to create municipalities, deployed in the law as mere boundary fixing. The Court laid down the "completeness test":

\begin{displayquote}
    Although Congress may delegate to another branch of the Government the power to fill in the details in the execution, enforcement or administration of a law, it is essential, to forestall a violation of the principle of separation of powers, that said law: (a) be complete in itself — it must set forth therein the policy to be executed, carried out or implemented by the delegate — and (b) fix a standard — the limits of which are sufficiently determinate or determinable — to which the delegate must conform in the performance of his functions. Indeed, without a statutory declaration of policy, the delegate would in effect, make or formulate such policy, which is the essence of every law; and, without the aforementioned standard, there would be no means to determine, with reasonable certainty, whether the delegate has acted within or beyond the scope of his authority. Hence, he could thereby arrogate upon himself the power, not only to make the law, but, also — and this is worse — to unmake it, by adopting measures inconsistent with the end sought to be attained by the Act of Congress, thus nullifying the principle of separation of powers and the system of checks and balances, and, consequently, undermining the very foundation of our Republican system.
\end{displayquote}

Under later elaborations of this test, the the law must be complete in all its terms and conditions when it leaves the legislative such that when it reaches the delegate the only thing he will have to do is to enforce it. \footnote{\textit{People v. Vera}, 65 Phil. 56, 115, citing 6, R.C.L., p. 165}

\textit{Edu v. Ericta},\footnote{G.R. No. L-32096, 1970.} provides additional indicia of completeness: 

\begin{displayquote}
    The test is the completeness of the statute in all its term and provisions when it leaves the hands of the legislature. To determine whether or not there is an undue delegation of legislative power, the inquiry must be directed to the scope and definiteness of the measure enacted. The legislature does not abdicate its functions when it describes what job must be done, who is to do it, and what is the scope of his authority.
\end{displayquote}

\begin{note}
    \textbf{See the work on incomplete contracts and see how they apply to laws. It is impossible for a legislator to see all contingencies, and every possible state of the world. Laws will always be incomplete.}
\end{note}

\textbf{Sufficient standards}  Under the sufficient standard test, there must be adequate guidelines or limitations in the law to map out the boundaries of the delegate's authority and prevent the delegation from running riot. \footnote{\textit{Tatad v. Secretary of Energy},G. R. No. 124360 (1997)} \textit{ABAKADA v. Ermita}  Delegation is valid when the law fixes a standard — the limits of which are sufficiently determinate and determinable — \enquote{to which the delegate must conform in the performance of his functions}.

\begin{displayquote}
A sufficient standard is one which defines legislative policy, marks its limits, maps out its boundaries and specifies the public agency to apply it. It indicates the circumstances under which the legislative command is to be effected. Both tests are intended to prevent a total transference of legislative authority to the delegate, who is not allowed to step into the shoes of the legislature and exercise a power essentially legislative.
\end{displayquote}

The Court has given Congress considerable leeway in defining these standards, often allowing broad language, or relying on an implied standard. \textit{Gancayco v. Quezon City} the Court struck down an action of the MMDA since there was no delegation at all. However, it upheld an ordinance of the Quezon City government, which was based on a delegated police power (under the city charter), to pass ordinances ``necessary and proper to provide
for the health and safety, promote the prosperity, improve the morals, peace, good order, comfort, and convenience of the city and the inhabitants thereof, and for the protection of property therein'' Decision indicates that government agencies can rely in laws other than the law delegating the power, including those subsequently passed, to provide a standard.\footnote{The Court adds: Corollarily, the policy of the Building Code, 2 82 8 which was passed after the Quezon City Ordinance, supports the purpose for the enactment of Ordinance No. 2904. The Building Code states: ``Section 102. Declaration of Policy. — It is hereby declared to be the policy of the State to safeguard life, health, property, and public welfare, consistent with the principles of sound environmental management and control;and to this end, make it the purpose of this Code to provide for all buildings and structures, a framework of minimum standards and requirements to regulate and control their location, site, design quality of materials, construction, occupancy, and maintenance.''}

In \textit{Chiongbian v. Orbos}, the Court maintained that the standard does not even have to be expressed in the statute under review, but may be gathered or implied.\footnote{Gathered seems to mean it can be assembled from other text that does not put forward a standard. Implied means concluded from some other express standard.} It can even be located in other statutes that have the same subject as the challenged legislation.\footnote{Ibid., citing Rabor v. Civil Service Commission, G.R. No. 111812, May 31, 1995.} In this case, the Court found sufficiently specific standard in the statute allowing for President's reorganization powers.


Allowance of broad standards is allowed for the same reason why delegation is allowed. \textit{Mistretta v. United States}\footnote{488 U.S. 361, 372(1989).} Acknowledged the necessity of “[I]n our increasingly complex society, replete with ever changing and more technical problems, Congress simply cannot do its job absent an ability to delegate power under broad general directives.” The standards which must accompany such a grant of power must not be unlimited, be unreasonable, or permit arbitrary action by the administrative body. (State v. Union Tank Car Co., 439 So. 2d 377 (La. 1983).)


\begin{quote}
The non-delegation doctrine does not prevent Congress from obtainiing assistance from coordinate Branches. The test of validity is that an "intelligble principle" must be established by the legislature where the agency of the delegated authority must adhere to specific directives that govern its authority. The delegation to the Commission was sufficiently detailed and specific to meet these requirements. The Commission was given substantial authority and discretion in setting these guidelines however, Congress established a classification hierarchy for federal crimes that the Commission was to use as an outline for its work.
\end{quote}


As long as Congress cannot abdicate from making the hard policy choices and leave both policy and implementation to the admin agencies. Congress can provide a broad policy, and the agency can fill in the blanks.

Jurisprudence is replete with instances where the Court upheld acts guided only by the barest, general guidance: "Simplicity and dignity" for the conduct of mandatory flag ceremonies in school;\footnote{Balbuena v. Secretary of Education, 110 Phil. 150 (1910).} "Public interest" in cancellation of permits to sell securities;\footnote{People v. Rosenthal, 68 Phil 328 (1939).}; More notoriously, the "interest of law and order" to justify the forced relocation of the Manguianes in Mindoro.\footnote{Rubi v. Provincial Board of Mindoro,39 Phil. 669 (1919).}
\newpage
\begin{shaded}
\textbf{Discussion Prompts for Chapter 1}
\begin{enumerate}
    \item Summarize and reflect on a case, concept or a rule found in this week’s readings. 
    \item According to Justice Moreland "the true distinction is between the delegation of power to make the law, which necessarily involves a discretion as to what it shall be, and conferring authority or discretion as to its execution, to be exercised under and in pursuance of the law. The first cannot be done; to the latter no valid objection can be made.” Can you come up with several valid objections?
    \item Compare the character or nature of the power delegated to the administrative agencies in Bowsher. Wayman, and Field. Are these different or the same?
    \item Assuming that the size and structure of Congress can be changed to scale, can it validly replace the rulemaking function of administrative agencies by just passing laws of sufficient specificity?
    \item Assuming that the size and structure of the Judiciary can be changed to scale, can it take over making the kind of rulings performed by administrative agencies?
\end{enumerate}


\end{shaded}


\end{document}
